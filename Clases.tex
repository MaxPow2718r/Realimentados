\documentclass[a4paper]{article}

\usepackage[utf8]{inputenc}

%url package
\usepackage[pdftex,colorlinks=true, urlcolor=blue, linkcolor = black, citecolor=black]{hyperref}

%math fonts
\usepackage{amsmath}
\usepackage{amsfonts}
\usepackage{amssymb}

%graphics enviroments
\usepackage{graphicx}
\usepackage{pgfplots}
\usepackage{import}
\usepackage{pdfpages}
\usepackage{transparent}
\pgfplotsset{compat=1.15}

%use inkscape to import directly svg images
\newcommand{\executeiffilenewer}[3]{%
		\ifnum\pdfstrcmp{\pdffilemoddate{#1}}%
				{\pdffilemoddate{#2}}>0%
		{\inmmediate\write18{#3}}\fi%
}
\newcommand{\includesvg}[1]{%
		\executeiffilenewer{#1.svg}{#1.pdf}%
		{inkscape -z -D --file=#1.svg %
		--export-pdf=#1.pdf --export-latex}%
		\input{#1.pdf_tex}%

}
%circuit diagrams packages
\usepackage{circuitikz}
\usetikzlibrary{shapes, arrows}

\begin{document}
\title{Control realimentado}
\maketitle
\section{Introducción}\label{sec:intro}
La ingeniería de control está presente en prácticamente todos los sistemas de ingeniería modernos.

El control es una tecnología que no está a la vista, está escondida.

\subsection{el control es la llave tecnológica que permite:}
\begin{itemize}
\item mejorar la calidad de productos
\item minimizar kas emisiones y los desechos.
\item protección del medio ambiente.
\end{itemize}

El control es una tarea multidisciplinaria que incluye:
\begin{itemize}
\item Sensores.
\item Actuadores.
\item Comunicación.
\item Computación.
\item Arquitectura.
\item Algoritmo.
\end{itemize}

\subsection{Objetivo}
El objetivo del diseño de control es lograr un nivel deseado de desempeño en la presencia de perturbaciones e incertidumbres.

\section{Modelado}\label{sec:modelado}
Para abordar el diseño de sistemas de control primero tenemos que entender como funciona (opera) el proceso o sistema a controlar.

Entonces, típicamente describiremos el proceso de forma matemática, esto es, un modelo matemático del proceso.

En la conclusión del modelo se define primero las variables relevantes del proceso.

Entonces tenemos (ejemplo):
\begin{itemize}
\item $h*$: nivel de referencia del molde.
\item $h(t)$: nivel actual (nivel actual dentro del molde o tanque).
\item $v(t)$: posición de la válvula.
\item $\sigma(t)$: velocidad de fundición (del proceso).
\item $q_{in}(t)$: flujo de entrada.
\item $q_{out}(t)$: flujo de salida.
\end{itemize}

La física nos sugiere el nivel del proceso (tanque, molde) sea proporcional a la integral de la diferencia entre el flujo de entrada y salida, es:
\begin{equation}\label{eq:altura}
		h(t) = \int_{-\infty}^{t}q_{in} (\tau) - q_{out} (\tau) \mathrm{d}\tau
\end{equation}

Se asume por simplicidad que el tanque tiene sección transversal uno (unidad).

También se asume que $v(t)$ y $\sigma(t)$ están calibradas de manera que indican los flujos de entrada y salida respectivamente.

\begin{equation}\label{eq:valvula}
		v(t) = q_{in} (t)
\end{equation}

\begin{equation}\label{eq:velocidad}
		\sigma (t) = q_{out} (t)
\end{equation}

Entonces el modelo del proceso queda como:
\begin{equation}\label{eq:altura2}
		h(t) = \int_{-\infty}^{t}(v(\tau) - \sigma(\tau)) \mathrm{d}\tau
\end{equation}
La velocidad de función se puede medir con la bastante precisión, pero los sensores de nivel del tanque suelen ser propensos a \emph{ruido} de medición de alta frecuencia, esto lo introducimos al modelo como señal aditiva $\eta(t)$

Entonces tenemos:
\begin{equation}\label{eq:ruido}
h_{m} (t) = h(t) + \eta(t)
\end{equation}

Donde $h_m(t)$ es la medición de $h(t)$ con ruido. En la figura 1 se muestra el diagrama de bloques del modelo del proceso general y las mediciones.

Este es un modelo muy simple, pero nos presenta la esencia del problema.

\tikzstyle{block} = [draw, rectangle, minimum height=3em, minimum width=6em]
\tikzstyle{sum} = [draw, circle, node distance = 1cm]
\tikzstyle{input} = [coordinate]
\tikzstyle{output} = [coordinate]
\tikzstyle{pinstyle} = [pin edge={to-,thin,black}]

\begin{figure}[h!]
\centering
\begin{circuitikz}[american voltages]
\draw
(0,0) node[mixer] (m) {}
(m.1) node[inputarrow] {} node[above left] {+}
to[short] ++(-4,0) node[below, text width = 2.5cm] {flujo de entrada}
(m.3) to[twoport,t = $\displaystyle \int$, >] ++(3,0)
(m.4) node[inputarrow, rotate = 270] {} node[above right] {-} to[short, -*] ++(0,2)
to[short] ++(0,2) node[right, text width = 2.5cm] {flujo de salida (velocidad de fundición)}
to[short] ++(0,-2)
to[short] ++(-2,0)
node[inputarrow, rotate =180] {} node[above, text width = 3cm] {medida de la\\ velocidad del flujo}
;
\end{circuitikz}
\end{figure}

\end{document}

