\documentclass[a4paper]{article}

\usepackage[utf8]{inputenc}
\usepackage[spanish]{babel}

%url package
\usepackage[pdftex,colorlinks=true, urlcolor=blue, linkcolor = black, citecolor=black]{hyperref}

%math fonts
\usepackage{amsmath}
\usepackage{amsfonts}
\usepackage{amssymb}

%graphics enviroments
\usepackage{graphicx}

%use inkscape to import directly svg images
\newcommand{\executeiffilenewer}[3]{%
		\ifnum\pdfstrcmp{\pdffilemoddate{#1}}%
				{\pdffilemoddate{#2}}>0%
		{\inmmediate\write18{#3}}\fi%
}
\newcommand{\includesvg}[1]{%
		\executeiffilenewer{#1.svg}{#1.pdf}%
		{inkscape -z -D --file=#1.svg %
		--export-pdf=#1.pdf --export-latex}%
		\input{#1.pdf_tex}%

}
%circuit diagrams packages
\usepackage{circuitikz}
\usetikzlibrary{shapes, arrows, babel}

%page format
\usepackage[margin = 2cm, left = 2cm, includefoot]{geometry}

\begin{document}
\title{Control realimentado}
\date{}
\maketitle
\section{Introducción}\label{sec:intro}

La ingeniería de control está presente en prácticamente todos los sistemas de ingeniería modernos.

El control es una tecnología que no está a la vista, está escondida.

\subsection{el control es la llave tecnológica que permite:}
\begin{itemize}
\item Mejorar la calidad de productos.
\item Minimizar kas emisiones y los desechos.
\item Protección del medio ambiente.
\end{itemize}

El control es una tarea multidisciplinaria que incluye:
\begin{itemize}
\item Sensores.
\item Actuadores.
\item Comunicación.
\item Computación.
\item Arquitectura.
\item Algoritmo.
\end{itemize}

\subsection{Objetivo}

El objetivo del diseño de control es lograr un nivel deseado de desempeño en la presencia de perturbaciones e incertidumbres.

\section{Modelado}\label{sec:modelado}
\subsection{Modelado matemático}

Para abordar el diseño de sistemas de control primero tenemos que entender como funciona (opera) el proceso o sistema a controlar.

Entonces, típicamente describiremos el proceso de forma matemática, esto es, un modelo matemático del proceso.

En la conclusión del modelo se define primero las variables relevantes del proceso.

Entonces tenemos (ejemplo):
\begin{itemize}
\item $h^*$: nivel de referencia del molde.
\item $h(t)$: nivel actual (nivel actual dentro del molde o tanque).
\item $v(t)$: posición de la válvula.
\item $\sigma(t)$: velocidad de fundición (del proceso).
\item $q_{in}(t)$: flujo de entrada.
\item $q_{out}(t)$: flujo de salida.
\end{itemize}

La física nos sugiere el nivel del proceso (tanque, molde) sea proporcional a la integral de la diferencia entre el flujo de entrada y salida, es:
\begin{equation}\label{eq:altura}
		h(t) = \int_{-\infty}^{t}q_{in} (\tau) - q_{out} (\tau) \mathrm{d}\tau
\end{equation}

Se asume por simplicidad que el tanque tiene sección transversal uno (unidad).

También se asume que $v(t)$ y $\sigma(t)$ están calibradas de manera que indican los flujos de entrada y salida respectivamente.

\begin{equation}\label{eq:valvula}
		v(t) = q_{in} (t)
\end{equation}

\begin{equation}\label{eq:velocidad}
		\sigma (t) = q_{out} (t)
\end{equation}

Entonces el modelo del proceso queda como:
\begin{equation}\label{eq:altura2}
		h(t) = \int_{-\infty}^{t}(v(\tau) - \sigma(\tau)) \mathrm{d}\tau
\end{equation}

La velocidad de función se puede medir con la bastante precisión, pero los sensores de nivel del tanque suelen ser propensos a \emph{ruido} de medición de alta frecuencia, esto lo introducimos al modelo como señal aditiva $\eta(t)$

Entonces tenemos:
\begin{equation}\label{eq:ruido}
h_{m} (t) = h(t) + \eta(t)
\end{equation}

Donde $h_m(t)$ es la medición de $h(t)$ con ruido. En la figura~\ref{fig:modelo_simple} se muestra el diagrama de bloques del modelo del proceso general y las mediciones.

Este es un modelo muy simple, pero nos presenta la esencia del problema.


\begin{figure}[!ht]\centering
\begin{circuitikz}[american voltages]
\draw
(0,0) node[mixer] (m) {}
(m.1) node[inputarrow] {} node[right] {+}
to[short] ++(-4,0) node[below, text width = 2.5cm] {flujo de entrada}
(m.3) to[twoport,t = $\displaystyle \int$, >] ++(3,0)
to[short, -*] ++(1,0)
to[short] ++(2,0) node[inputarrow] {}
node[above, text width = 2.5cm] {nivel del tanque}
to[short] ++(-2,0)
to[short] ++(0,-2)
node[mixer, below] (m2) {}
(m2.4) node[inputarrow, rotate = 270] {} node [below] {+}
(m2.3) node[inputarrow, rotate = 180] {} node[left] {+} to[short] ++(2,0)
node[above] {ruido}
(m2.2) to[short] ++(0,-2)
to[short] ++(-4,0) node[inputarrow, rotate = 180] {}
node[below, text width = 2.5cm] {medida del nivel del tanque homólogo}
(m.4) node[inputarrow, rotate = 270] {} node[below] {-} to[short, -*] ++(0,2)
to[short] ++(0,2) node[right, text width = 2.5cm] {flujo de salida (velocidad de fundición)}
to[short] ++(0,-2)
to[short] ++(-2,0)
node[inputarrow, rotate =180] {} node[above, text width = 3cm] {medida de la\\ velocidad del flujo}
;
\end{circuitikz}
\caption{Diagrama de bloques del un modelo simple de la dinámica del nivel del tanque, sensores y actuadores.}
\label{fig:modelo_simple}
\end{figure}

\subsection{Feedback y feedforward}

En general veremos más adelante que la idea de central en el control es la \underline{inversión} esta idea se puede lograr mediante el uso de técnicas claves, feedback y feedforward. Estas herramientas son una solución elegante y robusta a muchos problemas de diseño de control. Para el ejemplo del caso del control de nivel del molde (tanque), el controlador de retroalimentación (feedback) más simple es una ganancia constante $K$ que impulsa (actúa) proporcionalmente al error entre el nivel de referencia, $h^*$ y la medición real del modelo $h_m(t)$.
\begin{equation}\label{eq:velocidad_proporcional}
		v(t) = K\left(h^* - h_m(t)\right)
\end{equation}

Para anticipar como un controlador de este tipo debe desempeñarse o funcionar, observamos que primero debe ocurrir una desviación (error o diferencia) entre el punto de referencia (set point) y la medida actual. Antes de que el controlador pueda reaccionar sabemos, sin embargo, que un cambio de los requerimientos de la velocidad de fundición (del proceso) requiere una modificación del punto de operación de ka válvula.

Por lo tanto, en lugar de que se haga un cambio en la velocidad de fundición, lo que produce un error en el nivel del molde, al cual el control realimentado reacciona para corregir, podemos mejorar la estrategia (del control) cambiando la posición de la válvula, pro activamente, a esto se le llama \underline{feedforward} (prealimentación).

Esto nos da la siguiente forma para el controlador.
\begin{equation}\label{eq:controlador_completo}
		v(t) = K(\underbrace{\left[h^* - h_m(t)\right]}_{feedback}  + \underbrace{\left[\frac{1}{K} \sigma(t) \right]}_{feedforward}  )
\end{equation}

Observación:
\[h(t) = \int_{0}^{t}(v(\tau) - \sigma(\tau)) \mathrm{d}\tau\]
\[v(t) = q_{in}(t) \qquad;\qquad \sigma(t) = q_{out} (t) \]
\[h_m(t) = h(t) + \eta(t)\]

El control realimentado más simple es una ganancia constante $K$, que maneja la válvula proporcionalmente al error $e(t)$.

Por intuición se espera que $h(t) \approx h^*$.

\[h(t) = \int_{0}^{t}\left(K\left( \left[h^* - h_m(\tau) \right] + \left[ \frac{1}{K} \sigma(\tau) \right] \right) -\sigma(\tau)\right)\mathrm{d}\tau\]

\[h(t) = \int_{-\infty}^{t}\left(K(h^* - h_m(\tau)) \right) \mathrm{d}\tau~;\qquad h_m(t) = h(t) + \eta(t)\]

\[h(t) = \int_{-\infty}^{t}\left(Kh^* - Kh(\tau) - K\eta(\tau) \right) \mathrm{d}\tau\]
\[h(t) = \int_{-\infty}^{t}Kh^*\mathrm{d}\tau - \int_{-\infty}^{t}Kh(\tau) \mathrm{d}\tau - \int_{-\infty}^{t}K\eta(\tau) \mathrm{d}\tau\]

Sea $h(t)$ una señal causal $\implies h(t) \exists \forall t\ge 0 $:
\[\implies h(t) = \int_{0}^{t}Kh^*\mathrm{d}\tau - \int_{0}^{t}Kh(\tau)\mathrm{d}\tau - \int_{0}^{t}K\eta(\tau)\mathrm{d}\tau\]

Entonces aquí tenemos un controlador que está constituido por una parte retroalimentada (feedback) y una parte de acción preventiva (feedforward) el segundo termino entrega la acción predecible necesaria para compensar los cambios en la velocidad de fundición (para el ejemplo) mientras que el primer término reacciona al error dado por la diferencia entre el valor referencial y el valor actual del nivel.

\begin{figure}[!h]
\centering
\begin{circuitikz}[american voltages, thick,scale=0.6]
\draw
(0,0) node[mixer] (m1) {}
(m1.1) node[inputarrow] {} node[right] {+}
to[short] ++(-2,0) node[above] {referencia $h^*$}
(m1.3) to[short] ++(2,0)
node[mixer, right] (m2) {}
(m2.1) node[inputarrow] {} node[right] {+} node[above left] {$e(t)$}
(m2.3) to[twoport, t = $K$, >] ++(3,0)
node[mixer, right] (m3) {}
(m3.1) node[inputarrow] {} node[right] {+}
to[short] ++(-0.5,0) node[] (flujo_in) {}
(m3.3) to[twoport, t = $\displaystyle \int$, >] ++(3,0)
to[short, -*] ++(0,0)
to[short] ++(2,0) node[inputarrow] {} node[above, text width = 2.5cm] {nivel del molde (tanque) salida}
node[below] {$h(t)$}
to[short] ++(-2,0)
to[short] ++(0,-2)
node[mixer, below] (m4) {} node[inputarrow, rotate = 270] {} node[below] {+}
(m4.3) node[inputarrow, rotate = 180] {} node[left] {+} node[below right] {$\eta(t)$}
to[short] ++(2,0) node[above] {ruido}
(m4.2) to[short] ++(0,-2) to[short] ++(-4,0)
node[] (medicion) {}
to[short] ++(4,0) -| (m1.2)
node[inputarrow, rotate = 90] {}
node[above] {-} node[below left] {$h_m(t)$}
(m3.4) node[inputarrow, rotate = 270] {} node[below] {-}
to[short, -*] ++(0,2) to[short] ++(0,2) node[right, text width = 2.5cm] {flujo de salida corresponde a la velocidad de fundición} node[below, left] {$\sigma(t)$}
to[short] ++(0,-2)
to[twoport, t = $K^{-1}$, >] node[above, text width = 2cm] {medición de velocidad de fundición} ++(-4,0)
to[short] (m2.4) node[inputarrow, rotate = 270] {} node[below] {+}
;
\draw node[below of= flujo_in, node distance = 2cm, text width = 2.5cm] (texto) {flujo de entrada desde la válvula de control};
\draw [->] (texto) -- (flujo_in);
\draw node[below of = medicion, node distance = 0.5cm, text width = 3cm] {medición del nivel en el molde};
\end{circuitikz}
\caption{Modelo simplificado del control de nivel con acción compensativa para $q_{out} (t)$}
\label{fig:sistema_control}
\end{figure}

La figura~\ref{fig:sistema_control} muestra el diagrama de bloques para el sistema de control.

\subsection{Un primer indicador de ``trade-off'' (compromiso)}

Para simular el desempeño (comportamiento dinámico) del lazo de control con $K = 1$ y $K = 5$. Encontramos que para la ganancia del control con $K$ pequeño resulta una respuesta lenta a los cambios en el nivel del molde. En cambio, para la ganancia de control grande $(K = 5)$ resulta una respuesta rápida, pero también se incrementan los efectos de ruido (medida de ruido) como se ve en el menor control de nivel en estado estacionario por los movimientos más significativos y agresivos de la válvula, entonces aquí pareciera que los requerimientos de desempeño están en conflicto uno con el otro, al menos en algún grado.

\begin{figure}[!h]
\centering
\includegraphics[width = 0.5\textwidth]{trade_off}
\caption{Un primer indicador de ``trade-off''}
\label{trade}
\end{figure}

En este punto, un ingeniero de control  que no tiene una buena formación formal en el diseño de sistemas de control tendría dificultades para evaluar si este conflicto es simplemente una consecuencia  de tener un controlador tan simple o si es algo fundamental.

¿Cuánto esfuerzo se debe hacer para encontrar un valor adecuado para $K$?

¿Debería hacerse más esfuerzo en el modelar el proceso del nivel del molde?

En este curso iremos desarrollando los métodos para dar respuestas a estos y otras preguntas relacionadas. El ejemplo presentado nos motiva a formalizar la naturaleza del problema de control.
\newpage
\vspace{1mm}
\textbf{Definición 1}: \textit{El problema fundamental de control.}
\vspace{1mm}
\\
El problema de control de es encontrar una forma o camino técnicamente factible que actúa sobre un proceso dado de manera que el proceso se comporte lo más cerca posible a algún comportamiento deseado.\\
Más aún, este comportamiento deseado debe lograse (llevarse a cabo) con la presencia de incertidumbres del modelo (proceso) y ante la presencia de perturbaciones externas sin controlar (que no se manejan) que actúan sobre el proceso de esta definición se deduce o introducen las siguientes ideas.\\
\begin{enumerate}
		\item \textbf{\textit{Comportamiento deseado:}} este debe especificarse como parte del problema de diseño.
		\item \textbf{\textit{Factibilidad:}} esta significa que la solución debe satisfacer varias inversiones que pueden ser de naturaleza técnica, ambiental, económica u otra.
		\item \textbf{\textit{Incertidumbre:}} el conocimiento disponible sobre un sistema generalmente será limitado y de precisión limitada.
		\item \textbf{\textit{Acción:}} la solución requiere que la acción se aplique de alguna  manera al proceso, típicamente a través de más variables manipuladas que controlan a los actuadores.
		\item \textbf{\textit{Perturbaciones:}} el proceso que se controla generalmente tendrá otras entradas distintas de las que manipula el controlador. Estas son las perturbaciones.
		\item \textbf{\textit{Comportamiento aproximado:}} una solución factible rara vez será perfecta habrá inevitablemente un grado de aproximación para lograr el objetivo especificado.
		\item \textbf{\textit{Mediciones:}} estás mediciones son cruciales para que el controlador sepa que está haciendo realmente el sistema y cómo la afectan las perturbaciones inevitables.
\end{enumerate}

Desde ahora nos referiremos al proceso que se controla como ``la planta'' y diremos que \underline{la planta} se controla automáticamente cuando los objetivos de control se logran con una intervención humana poco frecuente.

\subsection{Solución al problema de control o inversión}

Una manera simple, pero que capta la esencia del problema de control, es a través de la \underline{inversión}. Esta idease basa en lo siguiente: \\
Conocemos que efecto produce una acción en la entrada de un sistema en la salida del sistema y además tenemos un comportamiento deseado para la salida, entonces se necesita invertir la solución entre la entrada y la salida para lograr es comportamiento deseado en la salida.

Esta idea aparentemente sencilla juega un rol profundo en el diseño del sistema de control. En el mundo real la mayoría de las dificultades del control se relacionan con encontrar una estrategia que \underline{capte la idea de inversión} y que al mismo tiempo consideren \underline{errores de modelado}, \underline{perturbaciones}, \underline{ruido de medición}, etc.

Supongamos que el comportamiento requerido está especificado por una señal objetivo (escalar) o referencia $r(t)$, para una variable de proceso particular $y(t)$ que tiene una perturbación aditiva $d(t)$. Además, tenemos disponible una sola variable manipulada, $u(t)$ (actuación).

Entonces, denotemos por ``$y$'' a una función temporal, es decir:

\[
		y = \left\{y(t) : t \in \mathbb{R}  \right\}
\]

La descripción de la solución para el problema de control veremos que es bastante general, que en principio puede aplicarse a los sistemas \underline{dinámicos no lineales}.

En particular, utilizaremos una función, $f\langle \circ \rangle$, para denotar un operador que ``mapea'' un espacio de funciones a otro.

Para esta interpretación general, introduciremos la siguiente notación. El símbolo ``$y$'' denotará un elemento de un espacio de funciones:

\[
		y = \left\{ y (t) : \mathbb{R} \to \mathbb{R}  \right\}
\]

Un operador $f\langle \circ \rangle $ representa un espacio de mapeo de sus funciones de $x$ en $x$, aquí podemos interpretar $f$ como una ganancia lineal que vincula un mínimo real, la entrada ``$u$'' con otra salida ``$y$''. Posteriormente se hará una interpretación más general que utiliza operadores dinámicos no lineales. Se asume que la salida está relacionada con la entrada por una función conocida por:

\begin{equation}\label{eq:mapeo}
		y = f\langle u \rangle + d
\end{equation}

Donde $f$ es una transformación o mapeo (posiblemente dinámico) que describe las relaciones de entrada-salida en la planta.

A la relación dada por la ecuación \ref{eq:mapeo} la llamaremos \underline{modelo}. Entonces, en el problema de control requiere que encontremos una forma de generar una ecuación de entrada de control ``$u$'', de tal modo que $y = r$ aplicando la idea de inversión para la solución, tenemos:

\begin{equation}\label{eq:referencia}
		y = r= f\langle u \rangle + d
\end{equation}

De esta podemos entonces, derivar, una \underline{ley de control}. Luego tenemos:

\[
		r = f\langle u \rangle +d \implies f\langle u \rangle = r - d
\]

\begin{equation}\label{eq:leydecontrol}
		u = f^{-1} \langle r-d \rangle \qquad (\mathrm{ley~de~control})
\end{equation}

La idea se ilustra en la figura \ref{fig:controlconceptual}.

\begin{figure}[!h]
\centering
\begin{circuitikz}
\draw
(0,0) node[mixer] (m1) {}
(m1.1) node[inputarrow] {} node[above left] {$r$} node[right] {+}
to[short] ++(-1,0)
(m1.3) to[short] ++(1,0) node[] (A) {} to[short] ++(0.5,0) to[twoport, text = $f^{-1} \langle u \rangle$, >] ++(1,0)
to[short] ++(0.5,0) node[] (B) {}
to[short] ++(0.5,0) node[above] {$u$}
to[short] ++(0.5,0)
to[short] ++(0.5,0) node[] (C) {}
to[twoport, text = $f\langle \circ \rangle$, >] ++(2,0) to[short] ++(1,0)
node[mixer, right] (m2) {}
(m2.1) node[inputarrow] {} node[right] {+}
(m2.3) to[short] ++(1,0) node[] (D) {}
to[short] ++(1,0) node[inputarrow] {} node[above right] {$y$}
(m2.4) node[inputarrow, rotate = 270] {} node[below] {+}
to[short,-*] ++(0,1)
to[short] ++(0,1) node[above right] {$d$}
to[short] ++(0,-1)
to[short] ++(-1,0) -| (m1.4) node[inputarrow, rotate = 270] {}
node[inputarrow, rotate = 270] {} node[below] {-}
;
\draw[dotted] ($(A.north)+(0,1)$) rectangle ($(B.south) + (0,-1)$);
\draw[dotted] ($(C.north) + (0,1)$) rectangle ($(D.south) + (0,-1)$);
\draw ($(A.south) + (1,-1.5)$) node[] {Controlador Conceptual};
\draw ($(C.south) + (2.5,-1.5)$) node[] {Planta};
\end{circuitikz}
\caption{Controlador conceptual}
\label{fig:controlconceptual}
\end{figure}

Esto es una solución conceptual del problema que al reflexionar sobre la ecuación \ref{eq:leydecontrol} se encuentra que debe cumplir ciertos requisitos estrictos para que funcione o sea exitosa. Esto para analizar las ecuaciones \ref{eq:referencia} y \ref{eq:leydecontrol} encontramos que se deben cumplir los siguientes requerimientos:

\begin{itemize}
		\item \textbf{R1:} - La transformación ``$f$'' debe describir exactamente la planta.
		\item \textbf{R2:} -``$f$'' debe estar bien formulada de manera que se tiene una salida acotada cuando la entrada es acotada, entonces decimos que la transformación \underline{``$f$'' es estable}.
		\item \textbf{R3:} - El inverso de $f$; $f^{-1}$, también debe estar bien formulado en el sentido de \textbf{R2}.
		\item \textbf{R4:} -La perturbación debe ser medible, de manera que ``$u$'' sea computable.
		\item \textbf{R5:} -La acción resultante (actuación) ``$u$'' debe ser \underline{realizable} y no violar ninguna restricción.
\end{itemize}

Estos son requisitos bastante exigentes que una parte importante de la teoría de control automático se ocupa del tema de como cambiar la arquitectura de control para que se logre la inversión, pero de una manera más robusta de forma que los estrictos requisitos vistos se puedan lograr.

Para ilustrar esto en la práctica

ilustracion bla bla

\end{document}
